\documentclass[a4paper,11pt]{article}
\usepackage[utf8]{inputenc}
\usepackage[T1]{fontenc}

\headsep1cm
\parindent0cm
\usepackage{amssymb, amstext, amsmath}
\usepackage{fancyhdr}
\usepackage{lastpage}
\usepackage{graphicx}

\lhead{\textbf{Electronic Market - Jason Implementation}}
\rhead{(Submission: 21.03.2013)}

\cfoot{}
\lfoot{Robert Schmidtke - F121550, Marco Eilers - F121763}
\rfoot{\thepage\ of \pageref{LastPage}}
\pagestyle{fancy}
\renewcommand{\footrulewidth}{0.4pt}

\setlength{\parskip}{4pt}

\begin{document}

\title{Multi-Agent Programming\\Assignment 3: Electronic Market - Jason Implementation}
\author{Robert Schmidtke - F121550, Marco Eilers - F121763}

\maketitle
\newpage

\section{Translation of Design into Implementation}
As in the previous assignment, we mostly referred to the GAIA model for the overall architecture of our solution, and to the Prometheus model for the details, since that is where the focus of the respective design methodologies is. All major components from our GAIA and Prometheus designs could be translated one to one to equivalent structures in Jason. Like in the design, our Jason implementation has two kinds of agents, the matchmaker and a number of traders. The environment contains a database which saves offers and requests, as specified in our GAIA model, and the negotiation works mostly the way we designed it in our Prometheus design. 

\subsection{Data Structures}
For the traders we had to translate their knowledge about all ongoing negotiations into beliefs. The most important ones are the following:
\begin{itemize}
  \item \texttt{lastPrice(Product,Partner,Price)} The last price an agent has offered to a partner in the negotiation about a specific product.
  \item \texttt{waitingFor(Product,Partner)} An agent should never make an offer for a specific product while it has not received a response to another offer it made for that product to a different trader, since this might result in it selling (or buying) the product twice if both offers are accepted. \texttt{waitingFor} is used to remember the fact that an agent is still waiting for a response from a specific partner concerning a specific product. If an agent is not waiting for anyone, \texttt{Partner} is \texttt{null}.
 \item \texttt{respondTo(Product,List)} The fact that agents wait for responses means that all offers must eventually get a response, since otherwise an agent might just stop all negotiations for a product. \texttt{respondTo} keeps track of a list of partners who all need to get a response concerning a product. Once an offer is accepted, all agents in this list will get a \texttt{reject} message.
\item \texttt{initialSent(Product,Partner)} It is possible that two agents simultaneously send each other an initial offer for a product. Since we want to avoid double negotiations, the buyer will ignore the incoming initial offer if it has already sent one itself. To keep track of this information, \texttt{initialSent} is used, which denotes that an agent has sent out an initial offer for a product to a partner.
\item \texttt{sales(Product,List)} stores a list of partners to who want to buy a product. The equivalent for the buyer is \texttt{negotiations(Product,List)}.
\item \texttt{offers(Product,MinPrice) } Indicates that this trader wants to sell a product for
at least MinPrice. The equivalent for the buyer is \texttt{requests(Product,MaxPrice)}.
\item \texttt{sold(Product)} This product has been sold, requests concerning it will be
rejected. The equivalent for the buyer is \texttt{bought(Product)}.
\end{itemize}
\noindent These data structures work fine for us, although it would have been nice to have more complex structures like records in some cases. That way, one could have saved all information concerning a negotiation with one partner in one object, instead of having several unconnected beliefs which together make up a conversation. 

\noindent The backing Java structures (\texttt{ItemDB} and \texttt{ItemDescriptor}) could be reused almost entirely. We only switched from describing items as a set of key-value-pairs of attributes to simply a list of attributes because this could easily be represented in AgentSpeak using lists. This also reduced the complexity of the configuration files and finding buyers and sellers for matching items in the \texttt{ItemDB}. Since pattern matching using the complete list of attributes for an item is performed, the UUID for each item descriptor was not necessary anymore to uniquely identify an item that is being negotiated.

The \texttt{ItemDB} is the only part of a custom environment, \texttt{ElectronicMarketEnv}. This environment is used only by the matchmaker agent by means of non-internal actions. The matchmaker itself only mitigates between trader agents and the \texttt{ItemDB}.

The previously introduced concept of initializing requests and offers for each trader using configuration files has been realized by defining a common base class for trader agents (\texttt{TraderAgClass}). When an agent's initialization is performed, a configuration file is used (determined by the agent's name) to register offers and requests with the matchmaker. Note that we shipped the same set of offers and requests as in the previous exercise.


\subsection{Ease of Implementation}
The implementation felt slightly more complicated than the JADE version, mostly because it took some time to get used to Jason's Prolog-like syntax and event-based programming model. One thing that troubled us for some time was Jason's parallel execution of several plans, especially in combination with its non-deterministim. The former caused quite a lot of errors in our earlier inplementations, since it meant that one could never assume that a certain belief existed, if there was any plan that removed and updated said belief, since it was always possible that other plans were executed between the removal of the belief and the storing of the updated version. The non-determinism then made it hard to reproduce those cases, so that proper debugging was almost impossible. We eventually solved this problem by marking all plans as \texttt{atomic}, so that once a plan is selected, only this plan will be followed until it is finished. Only then can other plans be executed. This solved a lot of problems that occured when incoming offers were processed \emph{while}, for example, initial offers were made.

We also had to make some compromises due to Jason's event based handling of goals, especially the fact that goals are forgotten once a plan to achieve them has been executed. Since, in some cases, we did not want the agent to forget its goals, we made them beliefs instead. One example is the fact that an agent wants to sell a product: This would normally be considered a goal, but since it is possible that buyers only enter the market place at a later time, we cannot allow the agent to forget about this goal once an initial plan has been followed. This is why we introduced the belief \texttt{offers(Product,MinPrice)} instead, which then creates other goals to start a negotiation, but which is still remembered if there are no buyers initially. This is why long-term goals are almost exclusively modelled as beliefs, whereas the steps to achieve these goals are usually modelled as goals.


\subsection{Changes to Design}
Since our negotiation algorithm was slightly underspecified in the original design, we reused the algorithm from our JADE implementation: If several partners compete for one product, the seller will always make a new offer to the partner with the lowest previous offer (and the other way round). Both buyers and sellers move twenty percent closer to their respective minimal/maximal price with each offer, until either the difference between two offers becomes less than 10 cent, or their own next offer would be worse than the partner's last offer. When one of these cases is reached, they accept the incoming offer if it is within their price limits and reject it otherwise.

\section{Notes}
The participants are specified as agents during system startup. They are given paths to initialization files from which they read their requested and offered items. This data is then used for registration with the matchmaking agent. The system of agents can be run through opening the \texttt{exercise3.mas2j} file in jEdit and then starting execution.

The execution of the negotiation process is very verbose: every negotiation initiation, received/accepted/rejected proposal as well as the end of a trade is logged to the console. It is thus possible to track all communication between agents during negotiation. The results of negotiations (successful or not) are printed with the final price. For each additional message (like registering offers and requests), an output line is produced as well.

\end{document}
