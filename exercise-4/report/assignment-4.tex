\documentclass[a4paper,11pt]{article}
\usepackage[utf8]{inputenc}
\usepackage[T1]{fontenc}

\headsep1cm
\parindent0cm
\usepackage{amssymb, amstext, amsmath}
\usepackage{fancyhdr}
\usepackage{lastpage}
\usepackage{graphicx}

\lhead{\textbf{Electronic Market - 2APL Implementation}}
\rhead{(Submission: 11.04.2013)}

\cfoot{}
\lfoot{Robert Schmidtke - F121550, Marco Eilers - F121763}
\rfoot{\thepage\ of \pageref{LastPage}}
\pagestyle{fancy}
\renewcommand{\footrulewidth}{0.4pt}

\setlength{\parskip}{4pt}

\begin{document}

\title{Multi-Agent Programming\\Assignment 4: Electronic Market - Jason Implementation}
\author{Robert Schmidtke - F121550, Marco Eilers - F121763}

\maketitle
\newpage

\section{Translation of Design into Implementation}
As in the previous assignment, we mostly referred to the GAIA model for the overall architecture of our solution, and to the Prometheus model for the details, since that is where the focus of the respective design methodologies is. All major components from our GAIA and Prometheus designs could be translated one to one to equivalent structures in 2APL. Like in the design, our 2APL implementation has two kinds of agents, the matchmaker and a number of traders. The environment contains a database which saves offers and requests, as specified in our GAIA model, and the negotiation works mostly the way we designed it in our Prometheus design.

The next section outlines a typical course of action of an agent in our system, whereas the following ones describe some details of our implementation.

\subsection{Typical agent lifecycle}
Each trader agent initially has a belief of the form \texttt{wantsToBuy(Product,Price)} or \texttt{wantsToSell(Product,Price)} (or both). Here, we will assume that our agent sells something, but the procedure for a buyer is mostly analogous. The initial goal \texttt{addGoals} generates a plan from a PG-rule which adds the goal \texttt{sold(Product,Price)}. A PG-rule then generates a plan from this goal which triggers all subsequent actions. First, the agent sends a message to the matchmaker to register its offer. The matchmaker saves this request in the \texttt{ItemDB}. After that, the agent queries the matchmaker for potential buyers. The matchmaker queries its database and sends a response containing the atom \texttt{setBuyers(Buyers)} to the trader, where \texttt{Buyers} is a list of other traders' names. The trader then creates some initial beliefs about the negotiations with those buyers and sends an initial offer of twice its minimal price to one of them using a message with the atom \texttt{reactToBuyOffer}, which contains information about the product in question and the offered price. The other trader uses a PC-rule to generate a plan to respond to this message. If a few conditions are fulfilled (e.g. the other trader is not currently waiting for a different trader's response concerning the product), this trader will call the PC-rule \texttt{respondToBuyOffer}. This rule generates a plan which decides if the agent can make a counteroffer. This is the case when the agent can go halfway from its last offer to its limit without making an offer that is worse for it then the other agent's offer. In this case it will simply respond with another message containing \texttt{reactToBuyOffer}. Otherwise, it checks if the other agent's offer is within its own limits, accepts if this is the case and rejects if not. In both cases it will send a message to its partner. In the former case, it will also inform the matchmaker that it no longer wants to buy the product, and creates the belief \texttt{bought(Product,Price)}, which achieves the initial goal. In the latter case, it looks through its list of potential sellers (if any) and makes a new offer to one of them .


\subsection{Data Structures}
The backing Java structures (\texttt{ItemDB} and \texttt{ItemDescriptor}) could be reused from the Jade assignment almost entirely, that is we only switched from describing items as a set of key-value-pairs of attributes to simply a list of attributes because this could easily be represented in 2APL using lists. This also reduced the complexity of the configuration files and finding buyers and sellers for matching items in the \texttt{ItemDB}. Since pattern matching using the complete list of attributes for an item is performed, the UUID for each item descriptor was not necessary anymore to uniquely identify an item that is being negotiated. Furthermore, we adapted the conversion mechanisms to and from \texttt{APLList}s when passing items to and from the environment.

The \texttt{ItemDB} is the only part of a custom environment, \texttt{ElectronicMarketEnv}. The environment is used only by the matchmaker agent by means of external actions. The matchmaker itself only mitigates between trader agents and the \texttt{ItemDB}.

Since there is no way of defining a common agent Java base class that would be able to parse initial requests and offers and add them to the goal base of each trading agent, we used the possibility of including beliefs with each agent in the \texttt{.mas} file by means of \texttt{<beliefs ... />} tags. These initial beliefs are then added as buy-/sell-goals by the initial goal \texttt{addGoals}. Note that we shipped the same set of offers and requests as in the previous exercise.

Each of our agents has the following beliefs for each product it wants to buy or sell:
\begin{itemize}
  \item \texttt{wantsToBuy(Product,MaxPrice)} This is the initial belief that an agent gets from an external file, and which is translated into a goal once the agent starts working. The equivalent for the seller is \texttt{wantsToSell}.
  \item \texttt{lastPrice(Product,Partner,Price)} The last price an agent has offered to a partner in the negotiation about a specific product.
  \item \texttt{waitingFor(Product,Partner)} An agent should never make an offer for a specific product while it has not received a response to another offer it made for that product to a different trader, since this might result in it selling (or buying) the product twice if both offers are accepted. \texttt{waitingFor} is used to remember the fact that an agent is still waiting for a response from a specific partner concerning a specific product. If an agent is not waiting for anyone, \texttt{Partner} is \texttt{null}.
 \item \texttt{respondTo(Product,List)} The fact that agents wait for responses means that all offers must eventually get a response, since otherwise an agent might just stop all negotiations for a product. \texttt{respondTo} keeps track of a list of partners who all need to get a response concerning a product. Once an offer is accepted, all agents in this list will get a \texttt{reject} message.
\item \texttt{initialSent(Product,Partner)} It is possible that two agents simultaneously send each other an initial offer for a product. Since we want to avoid double negotiations, the buyer will ignore the incoming initial offer if it has already sent one itself. To keep track of this information, \texttt{initialSent} is used, which denotes that an agent has sent out an initial offer for a product to a partner.
\item \texttt{sales(Product,List)} stores a list of partners to who want to buy a product. The equivalent for the buyer is \texttt{negotiations(Product,List)}.
\item \texttt{sold(Product,Price)} This product has been sold, requests concerning it will be
rejected. The equivalent for the buyer is \texttt{bought(Product,Price)}.
\end{itemize}

\noindent \texttt{BeliefUpdate}-rules are used to change these beliefs throughout an agent's life. In contrast to this, there are only three types of goals:

\begin{itemize}
  \item \texttt{sold(Product,MinPrice)} This is the main goal that each seller wants to reach for each item it sells. Agian, the equivalent for the buyer is \texttt{bought}.
  \item \texttt{addGoals} This initial goal stays active as long as the agent runs. For each \texttt{wantsToSell}-belief, it creates a \texttt{sold}-goal which is then pursued by the agent (and the same happens for items which the agent wants to buy).
\end{itemize}

\noindent The actions necessary to pursue these goals are implemented as PC-rules, since (1) they are not goals in and of themselves and (2) the procedure-call-like behaviour of these rules made it very easy to work with them. We did not use any PR-rules, since no plan in our system can actually permanently fail: Even if all your offers for a sale are rejected, it is always possible that another agent enters the market which offers what you want. It would have been possible to implement the reaction to a reject-message as a PR-rule, but we decided that getting rejections is normal behaviour and therefore there is no need to 'repair' anything.


\subsection{Ease of Implementation}
Figuring out the correct way to implement the environmental class was a little more complicated than in the previous assignments because the manual is only very vague in what class/interface should be extended/implemented. The documentation of the EIS was more helpful in this case, but we ended up disassembling the \texttt{blockworld.jar} example file (because of missing sources) and analyzed the byte code to find out what class should be used as a base class for our custom environment (the \texttt{apapl.Environment} class is fully functional, contrary to the \texttt{eis.EIDefaultImpl} class). Furthermore, the generation of the \texttt{MANIFEST.MF} file and the exhaustive inclusion of dependencies in the environment \texttt{.jar}-file were not straightforward since special export mechanisms and build path settings had to be used. 

The implementation of the agent's logic felt more straightforward than in the Jason exercise. The possibility to use \texttt{if-then-else} plans made made things much easier, since we did not need to write a new rule every time a decision has to be made (which was the case with Jason). Using \texttt{BeliefUpdate} rules was also more straightforward than updating beliefs manually every time, since they offer the possibility to react differently to different preconditions. This, too, was more complicated in Jason and usually meant that we had to write another rule every time a choice was made.

The implementation of goals felt more natural in 2APL than in Jason, because they are not forgotten once the initial event has been processed. This meant that we could actually have a goal \texttt{sold(Product,Price)} which would be active until a product had actually been sold. In Jason we had to represent this as a belief to make sure the knowledge was not lost.

\subsection{Changes to Design}
Since our negotiation algorithm was slightly underspecified in the original design, we reused the algorithm from our Jason implementation: If several partners compete for one product, the seller will always make a new offer to the partner with the lowest previous offer (and the other way round). Both buyers and sellers move fifty percent closer to their respective minimal/maximal price with each offer, until either the difference between two offers becomes less than 50 cent, or their own next offer would be worse than the partner's last offer. When one of these cases is reached, they accept the incoming offer if it is within their price limits and reject it otherwise.

\section{Notes}
The participants are specified as agents during system startup. They are given paths to initialization files from which they read their requested and offered items. This data is then used to set up buying and selling goals. The system of agents can be run through opening the \texttt{electronicmarket.mas} file in the 2APL platform and then starting execution.

The execution of the negotiation process is very verbose: every negotiation initiation, received/accepted/rejected proposal as well as the end of a trade is logged to the console. It is thus possible to track all communication between agents during negotiation. The results of negotiations (successful or not) are printed with the final price. For each additional message (like registering offers and requests), an output line is produced as well.

\end{document}
